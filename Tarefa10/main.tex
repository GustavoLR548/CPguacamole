\documentclass{article}
\usepackage[table]{xcolor}
\usepackage{listings}
%encoding
%--------------------------------------
\usepackage[T1]{fontenc}
\usepackage[utf8]{inputenc}
%--------------------------------------

%Portuguese-specific commands
%--------------------------------------
\usepackage[portuguese]{babel}
%--------------------------------------
\setlength{\arrayrulewidth}{0.5mm}
\setlength{\tabcolsep}{18pt}
\renewcommand{\arraystretch}{2.5}

\definecolor{codegreen}{rgb}{0,0.6,0}
\definecolor{codegray}{rgb}{0.5,0.5,0.5}
\definecolor{codepurple}{rgb}{0.58,0,0.82}
\definecolor{backcolour}{rgb}{0.95,0.95,0.92}

\lstdefinestyle{mystyle}{
    backgroundcolor=\color{backcolour},   
    commentstyle=\color{codegreen},
    keywordstyle=\color{magenta},
    numberstyle=\tiny\color{codegray},
    stringstyle=\color{codepurple},
    basicstyle=\ttfamily\footnotesize,
    breakatwhitespace=false,         
    breaklines=true,                 
    captionpos=b,                    
    keepspaces=true,                 
    numbers=left,                    
    numbersep=5pt,                  
    showspaces=false,                
    showstringspaces=false,
    showtabs=false,                  
    tabsize=2
}

\lstset{style=mystyle}

\begin{document} {

  \rowcolors{2}{gray!25}{white}

\section{Tabela}

\begin{table}[ht]
   \centering
	\begin{tabular}{ | c | c | c | }
   		\rowcolor{blue!80!yellow!50}
		\hline
		\multicolumn{3}{|c|}{Código sequencial} \\
		\rowcolor{green!80!yellow!50}
		\hline
		Contador & Sequencial & Paralelo \\
		Task-clock (msec) & 0,997 CPUs & 1,962 CPUs \\
		Cycles & 1,660 GHz   & 2,345 GHz \\
		L1-dcache-loads & 605,586 M/sec & 487,171 M/sec \\
		Instructions (per cycle) & 0,48 & 0,27 \\
    	LLC-load-misses & 4,67\% & 5,22\% \\
		Time Elapsed & 4,63 s & 2,93 s\\
		\hline
   \end{tabular}
   \caption{Comparação do sieve.c paralelo e sequencial no servidor parcode}
\end{table}

Professor, assim como discutido pelo teams: tanto o servidor parcode, quanto o meu computador não conseguiram extrair todas as informações usando o perf stats -d(sendo essas informações o  frontend cycles idle (ciclos ociosos na ULA) e o backend cycles idle (ciclos ociosos na busca de instrução)), logo essas informações foram substituidas pelo "Cycles" e o "L1-dcache-loads" respectivamente.

\newpage

\section{Código}

\lstinputlisting[language=c]{sieve.c}

\newpage

\section{Gargalos}

Usando o código na seção anterior, podemos identificar os gargalos nas regiões da linha 23 e 30, onde tem a maior parte da execução do programa, e é onde tem um laço de repetição 'for' dentro de outro 'for'

Para melhorar o tempo de execução do programa, o ideal seria a utilização do scheduler, pois irá garantir que haverá um maior balançeamento de cargas entre as threads durante a execução do programa, permitindo com que um número menor de threads fiquem ociosas.

\end{document}